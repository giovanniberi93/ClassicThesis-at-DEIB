% !TEX root = ../ClassicThesis_DEIB.tex

\chapter{Kinova Arm} \label{chap:kinovaArmChapter}

In this chapter, we'll describe our implementation of the action servers 

Kinova Arm chapter: SKETCH
\begin{itemize}
	\item moveit: cos'è, concetti di planning e execution
	\item scopo del braccio in GRAPE: scan della pianta, deployment, automa a stati che rappresenta la sequenza di azioni del braccio 
	\item problema 1 dovuto alla scarsa precisione: accenni alle parti di visual servo con marker per compensare questo problema in fase di grasping e deployment (cenni, fatta da Polito e Bascetta).
	\item problema 2 dovuto ai limiti di execution di moveit: impossibilità di fare un movimento di scansione fluido e completo usando moveit, metodo della jacobiana e la pubblicazione delle velocità
	\item utilizzo della camera per validazione del grasping e del deployment
	\item foto del braccio, immagini della nuvola di punti processata dal codice di ferran, alcune immagini dell'image processing nelle fasi di visual servo e di validazione di grasping/deployment
\end{itemize}