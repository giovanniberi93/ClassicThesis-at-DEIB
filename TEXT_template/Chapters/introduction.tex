% !TEX root = ../ClassicThesis_DEIB.tex

\chapter{Introduction} \label{chap:introduction}

In this thesis we worked on the design and development of the software components of a mobile manipulator for vineyard monitoring and protection, and on extensive validation of the whole platform during two \textit{on-the-field} sessions. The research and technology field this work belongs to is called called  \textit{agricultural robotics}, and it's oriented to the introduction of ICTs in an innovative way as solution for agricultural challenges. The successful introduction of these technique could gradually turn traditional farming into precision farming, with consequently decrease of the chemical load in food and environment, and improvement of profits and yield for farmers.

\par \ac{GRAPE} is an experimental project published by \ac{ECHORD++}, a robotic research project which aims to promote the interaction between robot manufacturers, researchers and users. More specifically, the goal of the platform developed in \ac{GRAPE} is to accomplish automatic distribution of synthetic pheromones for integrated pest control in vineyards.
The teams that carried out the whole system design, development and testing across the 18 months project duration was composed by both people from Politecnico, and people from Eurecat, a Catalan research center. 

The first, fundamental problem this thesis deals with is the development of an odometry and navigation systems for the Unmanned Ground Vehicle at issue; even if those problems could seem not so hard, they are significantly different to solve with respect to the correspondent indoor problems, because of some highly destabilizing factors due to the vineyard environment: bumpiness, steepness and consistency of the soil, time-varying weather conditions, unpredictable vegetation. The presence of these factors makes the \ac{LIDAR} sensor less reliable, and the pose estimation through wheel velocities almost completely untrustworthy.

The other main component of the platform treated in this thesis is the on-board manipulator, that realizes the actual deployment of the pheromone dispensers on the vines, and other tasks related to the selection of the most suitable point for deployment; the main difficulties on this topic were due to the motion limits of the manipulator itself, in terms of both accuracy and mobility.

\section{Thesis contribution}
To solve the problem of odometry estimation in such a complex environment like the vineyard, that presents large part of typical issues of outdoor robot navigation, we created a sensor fusion system, based on Robot Localization package, that exploits GPS, \ac{IMU}, and wheels sensed velocities to provide a very robust real-time estimate of the geolocalized pose and orientation of the platform. Since the traditional configuration of \ac{ROS} navigation stack for indoor navigation (\textit{i.e.} localization through \ac{AMCL} in a mapped environment) worked quite poorly in our context, mostly because of map repetitiveness, we crafted our own navigation configuration. In our settings, the obstacles considered by the robot are the ones, added by hand in a \textit{prohibition layer}, that represent the vineyard rows in both local and global costmap, and the ones detected by a frontal \ac{LIDAR} in the local costmap only. 
This configuration, tested with \textit{Timed Elastic Band} algorithm as local planner, demonstrated really good performances during \textit{on-the-field} sessions.
\par We also developed three different methods for the deploy



\section{Structure of the thesis}
This document is structured as follows.

\begin{itemize}
	\item In Chapter 2, we provide an overview of the main technological tool we adopted to develop out solution, together with the theoretical concepts required to better understand them. In particular, we give an overview of the functioning of \ac{ROS}, the framework we used to develop the entire platform, and of some \textit{off-the-shelf} packages we included in our project to solve sensor fusion, navigation, and manipulation problems.From the theoretical point of view, we explain basic concepts about pose estimation of the robot in the environment, and motion planning.
	\item In Chapter 3 we present a precise description the goals and the involved partners of \ac{GRAPE}, the experimental project this thesis is part of. Moreover, we present a collection of other European projects in agricultural robotics field in the last years.
	\item In Chapter 4 the description of the hardware and software architecture of the system implemented in the context of \ac{GRAPE} project. In the Section dedicated to hardware architecture, we provide technical specifications and justification for the choice of mobile base, robotic arm, and sensors mounted on the robot, while in the Section dedicated to software architecture we describe main software modules of the implementation, and the interfaces between them. Finally, we precisely specify the contribute of this thesis to the overall project.
	\item In Chapter 5, we present our solution to the problem of odometry estimation and autonomous navigation in the \ac{UGV} at issue. Since both of them went through several stages, for each significant step we highlight the weaknesses that pushed for search of better solutions, until the final configurations.
	\item In Chapter 6 we firstly list all the identified criticality of the used robotic arm, and the different motion techniques we used to overcome them. Secondly, we provide a fine-grained description of the manipulator specific tasks: the execution of the motion to properly scan the vine selected for deployment, different methods for the actual deployment of the dispenser, and validation of dispenser graping and deployment.
	\item In Chapter 7 we provide the results of extensive \textit{on-the-field} validation of sensor fusion, navigation, and manipulation systems, executed during the \textit{integration weeks}.
	\item In Chapter 8 we describe the configuration of Robì, a prototype small-sized mobile manipulator for agricultural application, as an alternative to Clearpath Husky as \ac{UGV} base for \ac{GRAPE} project.
	\item In Chapter 9 we draw our conclusions about our work, and we propose some possible future improvements to the platform.

\end{itemize}











