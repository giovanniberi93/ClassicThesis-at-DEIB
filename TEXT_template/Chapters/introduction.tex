% !TEX root = ../ClassicThesis_DEIB.tex

\chapter{Introduction} \label{chap:introduction}

In this thesis we worked on the design and development of the software components of a mobile manipulator for vineyard monitoring and protection, and on extensive validation of the whole platform during two \textit{on-the-field} sessions. The research and technology field this work belongs to is called called  \textit{agricultural robotics}, and it's oriented to the introduction of ICTs in an innovative way as solution for agricultural challenges. The successful introduction of these technique could gradually turn traditional farming into precision farming, with consequently decrease of the chemical load in food and environment, and improvement of profits and yield for farmers.

\par \ac{GRAPE} is an experimental project published by \ac{ECHORD++}, a robotic research project which aims to promote the interaction between robot manufacturers, researchers and users. More specifically, the goal of the platform developed in \ac{GRAPE} was to accomplish automatic distribution of synthetic pheromones for integrated pest control in vineyards.
The teams that carried out the whole system design, development and testing across the 18 months project duration was composed by both people from Politecnico, and people from Eurecat, a Catalan research center. 

The first, fundamental problem this thesis deals with is the development of an odometry and navigation systems for the Unmanned Ground Vehicle at issue; even if those problems could seem not so hard, they are significantly different to solve with respect to the correspondant indoor problems, because of some highly destabilizing factors due to the vineyard environment: bumpiness, steepness and consistency of the soil, time-varying weather conditions, unpredictable vegetation. The presence of these factors makes the \ac{LIDAR} sensor less reliable, and the pose estimation through wheel velocities almost completely untrustworthy.

The other main component of the platform treated in this thesis is the on-board manipulator, that realizes the actual deployment of the pheromone dispensers on the vines, and other tasks related to the selection of the most suitable point for deloyment; the main difficulties on this topic were due to the motion limits of the manipulator itself, in terms of both accuracy and mobility.


\section{Thesis contribution}

\section{Structure of the thesis}