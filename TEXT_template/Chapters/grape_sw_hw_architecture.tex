% !TEX root = ../ClassicThesis_DEIB.tex

\chapter{Grape hardware and software Architecure} \label{chap:grapeSoftwareArchitecture}

Grape architecture SKETCH:

\begin{itemize}
\item Hardware del robot:
		\begin{itemize}
			\item scelta dell'husky (adatto per rugged terrain, supporto di pacchetti ROS), foto
			\item scelta del braccio, con pro (compatto, 6 dof, power consumption non elevata) e contro (movimenti non precisi, fragile, scarso controllo di collisione)
			\item lidar sul braccio per scan
			\item lidar davanti per navigazione e eventualmente localizzazione
			\item camera montata sul braccio
			\item velodyne davanti (a cosa serve?)
			\item camera fissa per eventualmente scattare foto alla vigna
			\item imu
		\end{itemize}
	\item foto del robot 
	%(http://www.grape-project.eu/wp-content/uploads/2017/11/20171030-MacchineTrattori.pdf e eventuali altre che faremo)

\item Software del robot
	\begin{itemize}
		\item descrizione più precisa di quale è nel complesso la procedura: viene dato a mano (interfaccia grafica di Eurecat) una posizione finale, e serie di waypoints intermedi. Procedura di scan che usando PCL trova punti adatti al deployment. Procedura di deploymnet. Waypoint successivo.
		\item struttura col coordinatore, che chiama le 3 action (move\_base, scan, deploy)
		\item di ogni azione, il file che descrive i tipi di goal/result/feedback, con spiegazioni del significato dei parametri
	\end{itemize}
\end{itemize}