% ****************************************************************
% classicthesis-config.tex 
% formerly known as loadpackages.sty, classicthesis-ldpkg.sty, and
% classicthesis-preamble.sty Use it at the beginning of your
% ClassicThesis.tex, or as a LaTeX Preamble in your
% ClassicThesis.{tex,lyx} with % ****************************************************************
% classicthesis-config.tex 
% formerly known as loadpackages.sty, classicthesis-ldpkg.sty, and
% classicthesis-preamble.sty Use it at the beginning of your
% ClassicThesis.tex, or as a LaTeX Preamble in your
% ClassicThesis.{tex,lyx} with % ****************************************************************
% classicthesis-config.tex 
% formerly known as loadpackages.sty, classicthesis-ldpkg.sty, and
% classicthesis-preamble.sty Use it at the beginning of your
% ClassicThesis.tex, or as a LaTeX Preamble in your
% ClassicThesis.{tex,lyx} with % ****************************************************************
% classicthesis-config.tex 
% formerly known as loadpackages.sty, classicthesis-ldpkg.sty, and
% classicthesis-preamble.sty Use it at the beginning of your
% ClassicThesis.tex, or as a LaTeX Preamble in your
% ClassicThesis.{tex,lyx} with \input{classicthesis-config}
% ****************************************************************
% If you like the classicthesis, then I would appreciate a
% postcard. My address can be found in the file
% ClassicThesis.pdf. A collection of the postcards I received so
% far is available online at http://postcards.miede.de
% ****************************************************************

% ****************************************************************
% 1. Configure classicthesis for your needs here, e.g., remove
% "drafting" below in order to deactivate the time-stamp on the
% pages
% ****************************************************************

% !TEX root = ClassicThesis_DEIB.tex
\PassOptionsToPackage{eulerchapternumbers,
					listings,
				 	pdfspacing,
					floatperchapter,
					%linedheaders,
				 	subfig,
					beramono,
					eulermath,
					parts}{classicthesis}										

% miei packages
%\usepackage{graphicx}
%\usepackage{subcaption}

\usepackage[colorinlistoftodos]{todonotes}
\usepackage{pgf}
\usepackage{pgfplots}
\usepackage{tikz}
\usetikzlibrary{shapes,shapes.geometric,arrows,fit,calc,positioning,automata}
\usepackage[T1]{fontenc}
\usepackage{pifont}
% pseudocode
\usepackage{algorithm}
\usepackage[noend]{algpseudocode}
\makeatletter
\def\BState{\State\hskip-\ALG@thistlm}
\makeatother



% miei colori
\definecolor{asparagus}{rgb}{0.53, 0.66, 0.42}
\definecolor{camel}{rgb}{0.76, 0.6, 0.42}
\definecolor{bananayellow}{rgb}{1.0, 0.88, 0.21}
\definecolor{palette1}{rgb}{0.73,0.31,0.31}
\definecolor{palette2}{rgb}{0.85,0.61,0.63}
\definecolor{palette3}{rgb}{0.60,0.72,0.49}
\definecolor{palette4}{rgb}{0.37,0.47,0.36}
\definecolor{palette5}{rgb}{0.76,0.76,0.76}

% mie frecce

\makeatletter
\newcommand{\xdashrightarrow}[2][]{\ext@arrow 0359\rightarrowfill@@{#1}{#2}}
\newcommand{\xdashleftarrow}[2][]{\ext@arrow 3095\leftarrowfill@@{#1}{#2}}
\newcommand{\xdashleftrightarrow}[2][]{\ext@arrow 3359\leftrightarrowfill@@{#1}{#2}}
\def\rightarrowfill@@{\arrowfill@@\relax\relbar\rightarrow}
\def\leftarrowfill@@{\arrowfill@@\leftarrow\relbar\relax}
\def\leftrightarrowfill@@{\arrowfill@@\leftarrow\relbar\rightarrow}
\def\arrowfill@@#1#2#3#4{%
  $\m@th\thickmuskip0mu\medmuskip\thickmuskip\thinmuskip\thickmuskip
   \relax#4#1
   \xleaders\hbox{$#4#2$}\hfill
   #3$%
}
\makeatother
% ****************************************************************
% Triggers for this config
% **************************************************************** 
\usepackage{ifthen}
\newboolean{enable-backrefs} % enable backrefs in the bibliography
\setboolean{enable-backrefs}{false} % true false
% TODO backref is incompatible?
% ****************************************************************


% ****************************************************************
% 2. Personal data and user ad-hoc commands
% ****************************************************************
\newcommand{\myTitle}{A template for master thesis at DEIB\xspace}
\newcommand{\myTitleIT}{\textsc{Design, development\\and deployment\\of a mobile manipulator\\for vineyard monitoring\\and protection \xspace}}
\newcommand{\myFirstAuthorName}{Giovanni Beri\xspace}
\newcommand{\myMatrFirstAuthor}{852984\xspace}
\newcommand{\mySecondAuthorName}{Emanuele Mason\xspace}
\newcommand{\myMatrSecondAuthor}{222222\xspace}
\newcommand{\mySupervisor}{Prof. Matteo Matteucci\xspace} % relatore
\newcommand{\myOtherSupervisor}{Prof. Luca Bascetta\xspace} % co relatori
\newcommand{\myOtherOtherSupervisor}{Prof. my ASP supervisor\xspace}
\newcommand{\myCoExaminer}{Prof. Rodolfo Soncini-Sessa\xspace} % contro-relatore
\newcommand{\myFaculty}{Facolt\`a di Ingegneria\xspace}
\newcommand{\mySchool}{Scuola di Ingegneria Industriale e dell'Informazione\xspace}
\newcommand{\myDepartment}{Dipartimento di Elettronica, Informazione e Bioingegneria\xspace}
\newcommand{\myCourseFirstPart}{Master of Science in\xspace}
\newcommand{\myCourseFirstPartIT}{Corso di Laurea Magistrale in\xspace}
\newcommand{\myCourseSecondPart}{Computer Science and Engineering\xspace}
\newcommand{\myCourseSecondPartIT}{Ingegneria Informatica\xspace}
\newcommand{\myUni}{Politecnico di Milano\xspace}
\newcommand{\myLocation}{Milan\xspace}
\newcommand{\myTime}{2013\xspace}
\newcommand{\myVersion}{version 1.0\xspace}
\newcommand{\myAcademicYear}{Academic Year 2016-2017\xspace}
\newcommand{\myAcademicYearIT}{Anno Accademico 2016-2017\xspace}

% ********************************************************************
% Setup, fine tuning, and useful commands
% ********************************************************************

\newcounter{dummy} % necessary for correct hyperlinks (to index, bib, etc.)
\newlength{\abcd} % for ab..z string length calculation
\providecommand{\mLyX}{L\kern-.1667em\lower.25em\hbox{Y}\kern-.125emX\@}
% from here till the end of the section, you can modify whatever you want
\newcommand{\ie}{i.\,e.\ }
\newcommand{\Ie}{I.\,e.\ }
\newcommand{\eg}{e.\,g.\ }
\newcommand{\Eg}{E.\,g.\ }
% referencing commands
\newcommand{\myEq}[1]{equation \eqref{#1}}
\newcommand{\MyEq}[1]{Equation \eqref{#1}}
\newcommand{\myFig}[1]{figure \ref{#1}}
\newcommand{\MyFig}[1]{Figure \ref{#1}}
\newcommand{\myTab}[1]{table \ref{#1}}
\newcommand{\MyTab}[1]{Table \ref{#1}}
\newcommand{\mySubsec}[1]{subsection \ref{#1}}
\newcommand{\MySubsec}[1]{Subsection \ref{#1}}
\newcommand{\mySec}[1]{section \ref{#1}}
\newcommand{\MySec}[1]{Section \ref{#1}}
\newcommand{\myChap}[1]{chapter \ref{#1}}
\newcommand{\MyChap}[1]{Chapter \ref{#1}}
\newcommand{\myAppendix}[1]{appendix \ref{#1}}
\newcommand{\MyAppendix}[1]{Appendix \ref{#1}}
\newcommand{\myEmph}[1]{\textsc{#1}}

% **********************************************************************


% **********************************************************************
% 3. Loading some handy packages
% **********************************************************************
\PassOptionsToPackage{T1}{fontenc} % T2A for cyrillics
	\usepackage{fontenc} 
\PassOptionsToPackage{utf8}{inputenc} % latin9 (ISO-8859-9) = latin1+"Euro sign"
	\usepackage{inputenc}				

\PassOptionsToPackage{autostyle,italian=guillemets,threshold=2}{csquotes}
 	\usepackage{csquotes}

\PassOptionsToPackage{american,italian}{babel}
	 \usepackage{babel}

\usepackage{textcomp}% fix warning with missing font shapes
\usepackage{scrhack}% fix warnings when using KOMA with listings package          
\usepackage{xspace}% to get the spacing after macros right  
\usepackage{mparhack}% get marginpar right
\usepackage{fixltx2e}% fixes some LaTeX stuff
\usepackage{microtype}
\usepackage[normalem]{ulem} % to have strikethrough text

\PassOptionsToPackage{printonlyused,smaller}{acronym}
	\usepackage{acronym} % nice macros for handling all acronyms in the thesis
% **********************************************************************

%*********************************************************************************
% 3.a Math
%*********************************************************************************
\PassOptionsToPackage{fleqn}{amsmath} % math environments and more by the AMS 
 	\usepackage{amsmath}
 
\usepackage{amsthm}
\usepackage{amssymb}
\renewcommand\qedsymbol{$\blacksquare$}

\theoremstyle{definition}
\newtheorem{definition}{Definition}[chapter]
\newtheorem{theorem}{Theorem}[chapter]

\theoremstyle{plain}
\newtheorem{observation}[definition]{Observation}
\newtheorem{corollary}[theorem]{Corollary}
\newtheorem{lemma}[theorem]{Lemma}
% **********************************************************************


% **********************************************************************
% 4. Setup floats: tables, (sub)figures, and captions
% **********************************************************************
\usepackage{tabularx} % better tables
	\setlength{\extrarowheight}{3pt} % increase table row height
\newcommand{\tableheadline}[2]{\multicolumn{1}{#1}{\normalsize\spacedlowsmallcaps{#2}}}
\newcommand{\tableheadlineMore}[3]{\multicolumn{#1}{#2}{\normalsize\spacedlowsmallcaps{#3}}}
\newcommand{\tablefirstcol}[2]{\multicolumn{1}{#1}{\textbf{#2}}}

\usepackage{caption}
	\captionsetup{format=hang,labelfont={sf,bf},font=small}
\usepackage{colortbl}
\usepackage{multirow}

\usepackage{subfig}
\usepackage{siunitx}
% *********************************************************************

% *********************************************************************
% 5. Setup code listings
% *********************************************************************
\usepackage{listings}
\lstloadlanguages{bash, C++, Java, Matlab}

% for special keywords
\lstset{language=[LaTeX]Tex,
    keywordstyle=\color{RoyalBlue},%\bfseries,
    basicstyle=\small\ttfamily,
    %identifierstyle=\color{NavyBlue},
    commentstyle=\color{Green}\ttfamily,
    stringstyle=\rmfamily,
    numbers=none,%left,%
    numberstyle=\scriptsize,%\tiny
    stepnumber=5,
    numbersep=8pt,
    showstringspaces=false,
    breaklines=true,
    frameround=ftff,
    frame=single,
    belowcaptionskip=.75\baselineskip
    %frame=L
} 
% *********************************************************************


% *********************************************************************
% 6. PDFLaTeX, hyperreferences and citation backreferences
% *********************************************************************
% ********************************************************************
% Using PDFLaTeX
% ********************************************************************
\PassOptionsToPackage{pdftex,hyperfootnotes=false,pdfpagelabels}{hyperref}
	\usepackage{hyperref}  % backref linktocpage pagebackref
\pdfcompresslevel=9
\pdfadjustspacing=1 
\PassOptionsToPackage{pdftex}{graphicx}
	\usepackage{graphicx} 

% ********************************************************************
% Setup the style of the backrefs from the bibliography
% (translate the options to any language you use)
% ********************************************************************
\newcommand{\backrefnotcitedstring}{\relax}%(Not cited.)
\newcommand{\backrefcitedsinglestring}[1]{(Cited on page~#1.)}
\newcommand{\backrefcitedmultistring}[1]{(Cited on pages~#1.)}
\ifthenelse{\boolean{enable-backrefs}}%
{%
		\PassOptionsToPackage{hyperpageref}{backref}
		\usepackage{backref} % to be loaded after hyperref package 
		   \renewcommand{\backreftwosep}{ and~} % separate 2 pages
		   \renewcommand{\backreflastsep}{, and~} % separate last of longer list
		   \renewcommand*{\backref}[1]{}  % disable standard
		   \renewcommand*{\backrefalt}[4]{% detailed backref
		      \ifcase #1 %
		         \backrefnotcitedstring%
		      \or%
		         \backrefcitedsinglestring{#2}%
		      \else%
		         \backrefcitedmultistring{#2}%
		      \fi}%
}{\relax}    


% ****************************************************************
% PDF/A compliance
% ****************************************************************
% TODO not working: requires downloading color specification file in a specific
% tex folder and other hacks I don't want to spend time with
% \usepackage[a-1b]{pdfx}

% ********************************************************************
% Hyperreferences
% ********************************************************************
\hypersetup{%
    %draft,	% = no hyperlinking at all (useful in b/w printouts)
    colorlinks=true, linktocpage=true, pdfstartpage=3, pdfstartview=FitV,%
    % uncomment the following line if you want to have black links (e.g., for printing)
    %colorlinks=false, linktocpage=false, pdfborder={0 0 0}, pdfstartpage=3, pdfstartview=FitV,% 
    breaklinks=true, pdfpagemode=UseNone, pageanchor=true, pdfpagemode=UseOutlines,%
    plainpages=false, bookmarksnumbered, bookmarksopen=true, bookmarksopenlevel=1,%
    hypertexnames=true, pdfhighlight=/O,%nesting=true,%frenchlinks,%
    urlcolor=webbrown, linkcolor=RoyalBlue, citecolor=webgreen, %pagecolor=RoyalBlue,%
    %urlcolor=Black, linkcolor=Black, citecolor=Black, %pagecolor=Black,%
} 

    %pdftitle={\myTitle},%
    %pdfauthor={\textcopyright\ \myFirstAuthorName and \mySecondAuthorName, \myUni, \myFaculty},%
    %pdfsubject={},%
    %pdfkeywords={},%
    %pdfcreator={pdfLaTeX},%
    %pdfproducer={LaTeX with hyperref and classicthesis}%

%}   

% ********************************************************************
% Setup autoreferences
% ********************************************************************
% There are some issues regarding autorefnames
% http://www.ureader.de/msg/136221647.aspx
% http://www.tex.ac.uk/cgi-bin/texfaq2html?label=latexwords
% you have to redefine the makros for the 
% language you use, e.g., american, ngerman
% (as chosen when loading babel/AtBeginDocument)
% ********************************************************************
\makeatletter
\@ifpackageloaded{babel}%
    {%
       \addto\extrasamerican{%
					\renewcommand*{\figureautorefname}{Figure}%
					\renewcommand*{\tableautorefname}{Table}%
					\renewcommand*{\partautorefname}{Part}%
					\renewcommand*{\chapterautorefname}{Chapter}%
					\renewcommand*{\sectionautorefname}{Section}%
					\renewcommand*{\subsectionautorefname}{Section}%
					\renewcommand*{\subsubsectionautorefname}{Section}% 	
				}%
       \addto\extrasitalian{% 
					\renewcommand*{\paragraphautorefname}{Paragrafo}%
					\renewcommand*{\subparagraphautorefname}{Paragrafo}%
					\renewcommand*{\footnoteautorefname}{Nota a pié di pagina}%
					\renewcommand*{\FancyVerbLineautorefname}{Zeile}%
					\renewcommand*{\theoremautorefname}{Teorema}%
					\renewcommand*{\appendixautorefname}{Appendice}%
					\renewcommand*{\equationautorefname}{Equazione}%        
					\renewcommand*{\itemautorefname}{Punto}%
				}%	
			% Fix to getting autorefs for subfigures right (thanks to Belinda Vogt for changing the definition)
			\providecommand{\subfigureautorefname}{\figureautorefname}%  			
    }{\relax}
\makeatother

% ****************************************************************
% 7. Last calls before the bar closes
% ****************************************************************

\usepackage{classicthesis} 

% ****************************************************************
% ****************************************************************
% If you like the classicthesis, then I would appreciate a
% postcard. My address can be found in the file
% ClassicThesis.pdf. A collection of the postcards I received so
% far is available online at http://postcards.miede.de
% ****************************************************************

% ****************************************************************
% 1. Configure classicthesis for your needs here, e.g., remove
% "drafting" below in order to deactivate the time-stamp on the
% pages
% ****************************************************************

% !TEX root = ClassicThesis_DEIB.tex
\PassOptionsToPackage{eulerchapternumbers,
					listings,
				 	pdfspacing,
					floatperchapter,
					%linedheaders,
				 	subfig,
					beramono,
					eulermath,
					parts}{classicthesis}										

% miei packages
%\usepackage{graphicx}
%\usepackage{subcaption}

\usepackage[colorinlistoftodos]{todonotes}
\usepackage{pgf}
\usepackage{pgfplots}
\usepackage{tikz}
\usetikzlibrary{shapes,shapes.geometric,arrows,fit,calc,positioning,automata}
\usepackage[T1]{fontenc}
\usepackage{pifont}
% pseudocode
\usepackage{algorithm}
\usepackage[noend]{algpseudocode}
\makeatletter
\def\BState{\State\hskip-\ALG@thistlm}
\makeatother



% miei colori
\definecolor{asparagus}{rgb}{0.53, 0.66, 0.42}
\definecolor{camel}{rgb}{0.76, 0.6, 0.42}
\definecolor{bananayellow}{rgb}{1.0, 0.88, 0.21}
\definecolor{palette1}{rgb}{0.73,0.31,0.31}
\definecolor{palette2}{rgb}{0.85,0.61,0.63}
\definecolor{palette3}{rgb}{0.60,0.72,0.49}
\definecolor{palette4}{rgb}{0.37,0.47,0.36}
\definecolor{palette5}{rgb}{0.76,0.76,0.76}

% mie frecce

\makeatletter
\newcommand{\xdashrightarrow}[2][]{\ext@arrow 0359\rightarrowfill@@{#1}{#2}}
\newcommand{\xdashleftarrow}[2][]{\ext@arrow 3095\leftarrowfill@@{#1}{#2}}
\newcommand{\xdashleftrightarrow}[2][]{\ext@arrow 3359\leftrightarrowfill@@{#1}{#2}}
\def\rightarrowfill@@{\arrowfill@@\relax\relbar\rightarrow}
\def\leftarrowfill@@{\arrowfill@@\leftarrow\relbar\relax}
\def\leftrightarrowfill@@{\arrowfill@@\leftarrow\relbar\rightarrow}
\def\arrowfill@@#1#2#3#4{%
  $\m@th\thickmuskip0mu\medmuskip\thickmuskip\thinmuskip\thickmuskip
   \relax#4#1
   \xleaders\hbox{$#4#2$}\hfill
   #3$%
}
\makeatother
% ****************************************************************
% Triggers for this config
% **************************************************************** 
\usepackage{ifthen}
\newboolean{enable-backrefs} % enable backrefs in the bibliography
\setboolean{enable-backrefs}{false} % true false
% TODO backref is incompatible?
% ****************************************************************


% ****************************************************************
% 2. Personal data and user ad-hoc commands
% ****************************************************************
\newcommand{\myTitle}{A template for master thesis at DEIB\xspace}
\newcommand{\myTitleIT}{\textsc{Design, development\\and deployment\\of a mobile manipulator\\for vineyard monitoring\\and protection \xspace}}
\newcommand{\myFirstAuthorName}{Giovanni Beri\xspace}
\newcommand{\myMatrFirstAuthor}{852984\xspace}
\newcommand{\mySecondAuthorName}{Emanuele Mason\xspace}
\newcommand{\myMatrSecondAuthor}{222222\xspace}
\newcommand{\mySupervisor}{Prof. Matteo Matteucci\xspace} % relatore
\newcommand{\myOtherSupervisor}{Prof. Luca Bascetta\xspace} % co relatori
\newcommand{\myOtherOtherSupervisor}{Prof. my ASP supervisor\xspace}
\newcommand{\myCoExaminer}{Prof. Rodolfo Soncini-Sessa\xspace} % contro-relatore
\newcommand{\myFaculty}{Facolt\`a di Ingegneria\xspace}
\newcommand{\mySchool}{Scuola di Ingegneria Industriale e dell'Informazione\xspace}
\newcommand{\myDepartment}{Dipartimento di Elettronica, Informazione e Bioingegneria\xspace}
\newcommand{\myCourseFirstPart}{Master of Science in\xspace}
\newcommand{\myCourseFirstPartIT}{Corso di Laurea Magistrale in\xspace}
\newcommand{\myCourseSecondPart}{Computer Science and Engineering\xspace}
\newcommand{\myCourseSecondPartIT}{Ingegneria Informatica\xspace}
\newcommand{\myUni}{Politecnico di Milano\xspace}
\newcommand{\myLocation}{Milan\xspace}
\newcommand{\myTime}{2013\xspace}
\newcommand{\myVersion}{version 1.0\xspace}
\newcommand{\myAcademicYear}{Academic Year 2016-2017\xspace}
\newcommand{\myAcademicYearIT}{Anno Accademico 2016-2017\xspace}

% ********************************************************************
% Setup, fine tuning, and useful commands
% ********************************************************************

\newcounter{dummy} % necessary for correct hyperlinks (to index, bib, etc.)
\newlength{\abcd} % for ab..z string length calculation
\providecommand{\mLyX}{L\kern-.1667em\lower.25em\hbox{Y}\kern-.125emX\@}
% from here till the end of the section, you can modify whatever you want
\newcommand{\ie}{i.\,e.\ }
\newcommand{\Ie}{I.\,e.\ }
\newcommand{\eg}{e.\,g.\ }
\newcommand{\Eg}{E.\,g.\ }
% referencing commands
\newcommand{\myEq}[1]{equation \eqref{#1}}
\newcommand{\MyEq}[1]{Equation \eqref{#1}}
\newcommand{\myFig}[1]{figure \ref{#1}}
\newcommand{\MyFig}[1]{Figure \ref{#1}}
\newcommand{\myTab}[1]{table \ref{#1}}
\newcommand{\MyTab}[1]{Table \ref{#1}}
\newcommand{\mySubsec}[1]{subsection \ref{#1}}
\newcommand{\MySubsec}[1]{Subsection \ref{#1}}
\newcommand{\mySec}[1]{section \ref{#1}}
\newcommand{\MySec}[1]{Section \ref{#1}}
\newcommand{\myChap}[1]{chapter \ref{#1}}
\newcommand{\MyChap}[1]{Chapter \ref{#1}}
\newcommand{\myAppendix}[1]{appendix \ref{#1}}
\newcommand{\MyAppendix}[1]{Appendix \ref{#1}}
\newcommand{\myEmph}[1]{\textsc{#1}}

% **********************************************************************


% **********************************************************************
% 3. Loading some handy packages
% **********************************************************************
\PassOptionsToPackage{T1}{fontenc} % T2A for cyrillics
	\usepackage{fontenc} 
\PassOptionsToPackage{utf8}{inputenc} % latin9 (ISO-8859-9) = latin1+"Euro sign"
	\usepackage{inputenc}				

\PassOptionsToPackage{autostyle,italian=guillemets,threshold=2}{csquotes}
 	\usepackage{csquotes}

\PassOptionsToPackage{american,italian}{babel}
	 \usepackage{babel}

\usepackage{textcomp}% fix warning with missing font shapes
\usepackage{scrhack}% fix warnings when using KOMA with listings package          
\usepackage{xspace}% to get the spacing after macros right  
\usepackage{mparhack}% get marginpar right
\usepackage{fixltx2e}% fixes some LaTeX stuff
\usepackage{microtype}
\usepackage[normalem]{ulem} % to have strikethrough text

\PassOptionsToPackage{printonlyused,smaller}{acronym}
	\usepackage{acronym} % nice macros for handling all acronyms in the thesis
% **********************************************************************

%*********************************************************************************
% 3.a Math
%*********************************************************************************
\PassOptionsToPackage{fleqn}{amsmath} % math environments and more by the AMS 
 	\usepackage{amsmath}
 
\usepackage{amsthm}
\usepackage{amssymb}
\renewcommand\qedsymbol{$\blacksquare$}

\theoremstyle{definition}
\newtheorem{definition}{Definition}[chapter]
\newtheorem{theorem}{Theorem}[chapter]

\theoremstyle{plain}
\newtheorem{observation}[definition]{Observation}
\newtheorem{corollary}[theorem]{Corollary}
\newtheorem{lemma}[theorem]{Lemma}
% **********************************************************************


% **********************************************************************
% 4. Setup floats: tables, (sub)figures, and captions
% **********************************************************************
\usepackage{tabularx} % better tables
	\setlength{\extrarowheight}{3pt} % increase table row height
\newcommand{\tableheadline}[2]{\multicolumn{1}{#1}{\normalsize\spacedlowsmallcaps{#2}}}
\newcommand{\tableheadlineMore}[3]{\multicolumn{#1}{#2}{\normalsize\spacedlowsmallcaps{#3}}}
\newcommand{\tablefirstcol}[2]{\multicolumn{1}{#1}{\textbf{#2}}}

\usepackage{caption}
	\captionsetup{format=hang,labelfont={sf,bf},font=small}
\usepackage{colortbl}
\usepackage{multirow}

\usepackage{subfig}
\usepackage{siunitx}
% *********************************************************************

% *********************************************************************
% 5. Setup code listings
% *********************************************************************
\usepackage{listings}
\lstloadlanguages{bash, C++, Java, Matlab}

% for special keywords
\lstset{language=[LaTeX]Tex,
    keywordstyle=\color{RoyalBlue},%\bfseries,
    basicstyle=\small\ttfamily,
    %identifierstyle=\color{NavyBlue},
    commentstyle=\color{Green}\ttfamily,
    stringstyle=\rmfamily,
    numbers=none,%left,%
    numberstyle=\scriptsize,%\tiny
    stepnumber=5,
    numbersep=8pt,
    showstringspaces=false,
    breaklines=true,
    frameround=ftff,
    frame=single,
    belowcaptionskip=.75\baselineskip
    %frame=L
} 
% *********************************************************************


% *********************************************************************
% 6. PDFLaTeX, hyperreferences and citation backreferences
% *********************************************************************
% ********************************************************************
% Using PDFLaTeX
% ********************************************************************
\PassOptionsToPackage{pdftex,hyperfootnotes=false,pdfpagelabels}{hyperref}
	\usepackage{hyperref}  % backref linktocpage pagebackref
\pdfcompresslevel=9
\pdfadjustspacing=1 
\PassOptionsToPackage{pdftex}{graphicx}
	\usepackage{graphicx} 

% ********************************************************************
% Setup the style of the backrefs from the bibliography
% (translate the options to any language you use)
% ********************************************************************
\newcommand{\backrefnotcitedstring}{\relax}%(Not cited.)
\newcommand{\backrefcitedsinglestring}[1]{(Cited on page~#1.)}
\newcommand{\backrefcitedmultistring}[1]{(Cited on pages~#1.)}
\ifthenelse{\boolean{enable-backrefs}}%
{%
		\PassOptionsToPackage{hyperpageref}{backref}
		\usepackage{backref} % to be loaded after hyperref package 
		   \renewcommand{\backreftwosep}{ and~} % separate 2 pages
		   \renewcommand{\backreflastsep}{, and~} % separate last of longer list
		   \renewcommand*{\backref}[1]{}  % disable standard
		   \renewcommand*{\backrefalt}[4]{% detailed backref
		      \ifcase #1 %
		         \backrefnotcitedstring%
		      \or%
		         \backrefcitedsinglestring{#2}%
		      \else%
		         \backrefcitedmultistring{#2}%
		      \fi}%
}{\relax}    


% ****************************************************************
% PDF/A compliance
% ****************************************************************
% TODO not working: requires downloading color specification file in a specific
% tex folder and other hacks I don't want to spend time with
% \usepackage[a-1b]{pdfx}

% ********************************************************************
% Hyperreferences
% ********************************************************************
\hypersetup{%
    %draft,	% = no hyperlinking at all (useful in b/w printouts)
    colorlinks=true, linktocpage=true, pdfstartpage=3, pdfstartview=FitV,%
    % uncomment the following line if you want to have black links (e.g., for printing)
    %colorlinks=false, linktocpage=false, pdfborder={0 0 0}, pdfstartpage=3, pdfstartview=FitV,% 
    breaklinks=true, pdfpagemode=UseNone, pageanchor=true, pdfpagemode=UseOutlines,%
    plainpages=false, bookmarksnumbered, bookmarksopen=true, bookmarksopenlevel=1,%
    hypertexnames=true, pdfhighlight=/O,%nesting=true,%frenchlinks,%
    urlcolor=webbrown, linkcolor=RoyalBlue, citecolor=webgreen, %pagecolor=RoyalBlue,%
    %urlcolor=Black, linkcolor=Black, citecolor=Black, %pagecolor=Black,%
} 

    %pdftitle={\myTitle},%
    %pdfauthor={\textcopyright\ \myFirstAuthorName and \mySecondAuthorName, \myUni, \myFaculty},%
    %pdfsubject={},%
    %pdfkeywords={},%
    %pdfcreator={pdfLaTeX},%
    %pdfproducer={LaTeX with hyperref and classicthesis}%

%}   

% ********************************************************************
% Setup autoreferences
% ********************************************************************
% There are some issues regarding autorefnames
% http://www.ureader.de/msg/136221647.aspx
% http://www.tex.ac.uk/cgi-bin/texfaq2html?label=latexwords
% you have to redefine the makros for the 
% language you use, e.g., american, ngerman
% (as chosen when loading babel/AtBeginDocument)
% ********************************************************************
\makeatletter
\@ifpackageloaded{babel}%
    {%
       \addto\extrasamerican{%
					\renewcommand*{\figureautorefname}{Figure}%
					\renewcommand*{\tableautorefname}{Table}%
					\renewcommand*{\partautorefname}{Part}%
					\renewcommand*{\chapterautorefname}{Chapter}%
					\renewcommand*{\sectionautorefname}{Section}%
					\renewcommand*{\subsectionautorefname}{Section}%
					\renewcommand*{\subsubsectionautorefname}{Section}% 	
				}%
       \addto\extrasitalian{% 
					\renewcommand*{\paragraphautorefname}{Paragrafo}%
					\renewcommand*{\subparagraphautorefname}{Paragrafo}%
					\renewcommand*{\footnoteautorefname}{Nota a pié di pagina}%
					\renewcommand*{\FancyVerbLineautorefname}{Zeile}%
					\renewcommand*{\theoremautorefname}{Teorema}%
					\renewcommand*{\appendixautorefname}{Appendice}%
					\renewcommand*{\equationautorefname}{Equazione}%        
					\renewcommand*{\itemautorefname}{Punto}%
				}%	
			% Fix to getting autorefs for subfigures right (thanks to Belinda Vogt for changing the definition)
			\providecommand{\subfigureautorefname}{\figureautorefname}%  			
    }{\relax}
\makeatother

% ****************************************************************
% 7. Last calls before the bar closes
% ****************************************************************

\usepackage{classicthesis} 

% ****************************************************************
% ****************************************************************
% If you like the classicthesis, then I would appreciate a
% postcard. My address can be found in the file
% ClassicThesis.pdf. A collection of the postcards I received so
% far is available online at http://postcards.miede.de
% ****************************************************************

% ****************************************************************
% 1. Configure classicthesis for your needs here, e.g., remove
% "drafting" below in order to deactivate the time-stamp on the
% pages
% ****************************************************************

% !TEX root = ClassicThesis_DEIB.tex
\PassOptionsToPackage{eulerchapternumbers,
					listings,
				 	pdfspacing,
					floatperchapter,
					%linedheaders,
				 	subfig,
					beramono,
					eulermath,
					parts}{classicthesis}										

% miei packages
%\usepackage{graphicx}
%\usepackage{subcaption}

\usepackage[colorinlistoftodos]{todonotes}
\usepackage{pgf}
\usepackage{pgfplots}
\usepackage{tikz}
\usetikzlibrary{shapes,shapes.geometric,arrows,fit,calc,positioning,automata}
\usepackage[T1]{fontenc}
\usepackage{pifont}
% pseudocode
\usepackage{algorithm}
\usepackage[noend]{algpseudocode}
\makeatletter
\def\BState{\State\hskip-\ALG@thistlm}
\makeatother



% miei colori
\definecolor{asparagus}{rgb}{0.53, 0.66, 0.42}
\definecolor{camel}{rgb}{0.76, 0.6, 0.42}
\definecolor{bananayellow}{rgb}{1.0, 0.88, 0.21}
\definecolor{palette1}{rgb}{0.73,0.31,0.31}
\definecolor{palette2}{rgb}{0.85,0.61,0.63}
\definecolor{palette3}{rgb}{0.60,0.72,0.49}
\definecolor{palette4}{rgb}{0.37,0.47,0.36}
\definecolor{palette5}{rgb}{0.76,0.76,0.76}

% mie frecce

\makeatletter
\newcommand{\xdashrightarrow}[2][]{\ext@arrow 0359\rightarrowfill@@{#1}{#2}}
\newcommand{\xdashleftarrow}[2][]{\ext@arrow 3095\leftarrowfill@@{#1}{#2}}
\newcommand{\xdashleftrightarrow}[2][]{\ext@arrow 3359\leftrightarrowfill@@{#1}{#2}}
\def\rightarrowfill@@{\arrowfill@@\relax\relbar\rightarrow}
\def\leftarrowfill@@{\arrowfill@@\leftarrow\relbar\relax}
\def\leftrightarrowfill@@{\arrowfill@@\leftarrow\relbar\rightarrow}
\def\arrowfill@@#1#2#3#4{%
  $\m@th\thickmuskip0mu\medmuskip\thickmuskip\thinmuskip\thickmuskip
   \relax#4#1
   \xleaders\hbox{$#4#2$}\hfill
   #3$%
}
\makeatother
% ****************************************************************
% Triggers for this config
% **************************************************************** 
\usepackage{ifthen}
\newboolean{enable-backrefs} % enable backrefs in the bibliography
\setboolean{enable-backrefs}{false} % true false
% TODO backref is incompatible?
% ****************************************************************


% ****************************************************************
% 2. Personal data and user ad-hoc commands
% ****************************************************************
\newcommand{\myTitle}{A template for master thesis at DEIB\xspace}
\newcommand{\myTitleIT}{\textsc{Design, development\\and deployment\\of a mobile manipulator\\for vineyard monitoring\\and protection \xspace}}
\newcommand{\myFirstAuthorName}{Giovanni Beri\xspace}
\newcommand{\myMatrFirstAuthor}{852984\xspace}
\newcommand{\mySecondAuthorName}{Emanuele Mason\xspace}
\newcommand{\myMatrSecondAuthor}{222222\xspace}
\newcommand{\mySupervisor}{Prof. Matteo Matteucci\xspace} % relatore
\newcommand{\myOtherSupervisor}{Prof. Luca Bascetta\xspace} % co relatori
\newcommand{\myOtherOtherSupervisor}{Prof. my ASP supervisor\xspace}
\newcommand{\myCoExaminer}{Prof. Rodolfo Soncini-Sessa\xspace} % contro-relatore
\newcommand{\myFaculty}{Facolt\`a di Ingegneria\xspace}
\newcommand{\mySchool}{Scuola di Ingegneria Industriale e dell'Informazione\xspace}
\newcommand{\myDepartment}{Dipartimento di Elettronica, Informazione e Bioingegneria\xspace}
\newcommand{\myCourseFirstPart}{Master of Science in\xspace}
\newcommand{\myCourseFirstPartIT}{Corso di Laurea Magistrale in\xspace}
\newcommand{\myCourseSecondPart}{Computer Science and Engineering\xspace}
\newcommand{\myCourseSecondPartIT}{Ingegneria Informatica\xspace}
\newcommand{\myUni}{Politecnico di Milano\xspace}
\newcommand{\myLocation}{Milan\xspace}
\newcommand{\myTime}{2013\xspace}
\newcommand{\myVersion}{version 1.0\xspace}
\newcommand{\myAcademicYear}{Academic Year 2016-2017\xspace}
\newcommand{\myAcademicYearIT}{Anno Accademico 2016-2017\xspace}

% ********************************************************************
% Setup, fine tuning, and useful commands
% ********************************************************************

\newcounter{dummy} % necessary for correct hyperlinks (to index, bib, etc.)
\newlength{\abcd} % for ab..z string length calculation
\providecommand{\mLyX}{L\kern-.1667em\lower.25em\hbox{Y}\kern-.125emX\@}
% from here till the end of the section, you can modify whatever you want
\newcommand{\ie}{i.\,e.\ }
\newcommand{\Ie}{I.\,e.\ }
\newcommand{\eg}{e.\,g.\ }
\newcommand{\Eg}{E.\,g.\ }
% referencing commands
\newcommand{\myEq}[1]{equation \eqref{#1}}
\newcommand{\MyEq}[1]{Equation \eqref{#1}}
\newcommand{\myFig}[1]{figure \ref{#1}}
\newcommand{\MyFig}[1]{Figure \ref{#1}}
\newcommand{\myTab}[1]{table \ref{#1}}
\newcommand{\MyTab}[1]{Table \ref{#1}}
\newcommand{\mySubsec}[1]{subsection \ref{#1}}
\newcommand{\MySubsec}[1]{Subsection \ref{#1}}
\newcommand{\mySec}[1]{section \ref{#1}}
\newcommand{\MySec}[1]{Section \ref{#1}}
\newcommand{\myChap}[1]{chapter \ref{#1}}
\newcommand{\MyChap}[1]{Chapter \ref{#1}}
\newcommand{\myAppendix}[1]{appendix \ref{#1}}
\newcommand{\MyAppendix}[1]{Appendix \ref{#1}}
\newcommand{\myEmph}[1]{\textsc{#1}}

% **********************************************************************


% **********************************************************************
% 3. Loading some handy packages
% **********************************************************************
\PassOptionsToPackage{T1}{fontenc} % T2A for cyrillics
	\usepackage{fontenc} 
\PassOptionsToPackage{utf8}{inputenc} % latin9 (ISO-8859-9) = latin1+"Euro sign"
	\usepackage{inputenc}				

\PassOptionsToPackage{autostyle,italian=guillemets,threshold=2}{csquotes}
 	\usepackage{csquotes}

\PassOptionsToPackage{american,italian}{babel}
	 \usepackage{babel}

\usepackage{textcomp}% fix warning with missing font shapes
\usepackage{scrhack}% fix warnings when using KOMA with listings package          
\usepackage{xspace}% to get the spacing after macros right  
\usepackage{mparhack}% get marginpar right
\usepackage{fixltx2e}% fixes some LaTeX stuff
\usepackage{microtype}
\usepackage[normalem]{ulem} % to have strikethrough text

\PassOptionsToPackage{printonlyused,smaller}{acronym}
	\usepackage{acronym} % nice macros for handling all acronyms in the thesis
% **********************************************************************

%*********************************************************************************
% 3.a Math
%*********************************************************************************
\PassOptionsToPackage{fleqn}{amsmath} % math environments and more by the AMS 
 	\usepackage{amsmath}
 
\usepackage{amsthm}
\usepackage{amssymb}
\renewcommand\qedsymbol{$\blacksquare$}

\theoremstyle{definition}
\newtheorem{definition}{Definition}[chapter]
\newtheorem{theorem}{Theorem}[chapter]

\theoremstyle{plain}
\newtheorem{observation}[definition]{Observation}
\newtheorem{corollary}[theorem]{Corollary}
\newtheorem{lemma}[theorem]{Lemma}
% **********************************************************************


% **********************************************************************
% 4. Setup floats: tables, (sub)figures, and captions
% **********************************************************************
\usepackage{tabularx} % better tables
	\setlength{\extrarowheight}{3pt} % increase table row height
\newcommand{\tableheadline}[2]{\multicolumn{1}{#1}{\normalsize\spacedlowsmallcaps{#2}}}
\newcommand{\tableheadlineMore}[3]{\multicolumn{#1}{#2}{\normalsize\spacedlowsmallcaps{#3}}}
\newcommand{\tablefirstcol}[2]{\multicolumn{1}{#1}{\textbf{#2}}}

\usepackage{caption}
	\captionsetup{format=hang,labelfont={sf,bf},font=small}
\usepackage{colortbl}
\usepackage{multirow}

\usepackage{subfig}
\usepackage{siunitx}
% *********************************************************************

% *********************************************************************
% 5. Setup code listings
% *********************************************************************
\usepackage{listings}
\lstloadlanguages{bash, C++, Java, Matlab}

% for special keywords
\lstset{language=[LaTeX]Tex,
    keywordstyle=\color{RoyalBlue},%\bfseries,
    basicstyle=\small\ttfamily,
    %identifierstyle=\color{NavyBlue},
    commentstyle=\color{Green}\ttfamily,
    stringstyle=\rmfamily,
    numbers=none,%left,%
    numberstyle=\scriptsize,%\tiny
    stepnumber=5,
    numbersep=8pt,
    showstringspaces=false,
    breaklines=true,
    frameround=ftff,
    frame=single,
    belowcaptionskip=.75\baselineskip
    %frame=L
} 
% *********************************************************************


% *********************************************************************
% 6. PDFLaTeX, hyperreferences and citation backreferences
% *********************************************************************
% ********************************************************************
% Using PDFLaTeX
% ********************************************************************
\PassOptionsToPackage{pdftex,hyperfootnotes=false,pdfpagelabels}{hyperref}
	\usepackage{hyperref}  % backref linktocpage pagebackref
\pdfcompresslevel=9
\pdfadjustspacing=1 
\PassOptionsToPackage{pdftex}{graphicx}
	\usepackage{graphicx} 

% ********************************************************************
% Setup the style of the backrefs from the bibliography
% (translate the options to any language you use)
% ********************************************************************
\newcommand{\backrefnotcitedstring}{\relax}%(Not cited.)
\newcommand{\backrefcitedsinglestring}[1]{(Cited on page~#1.)}
\newcommand{\backrefcitedmultistring}[1]{(Cited on pages~#1.)}
\ifthenelse{\boolean{enable-backrefs}}%
{%
		\PassOptionsToPackage{hyperpageref}{backref}
		\usepackage{backref} % to be loaded after hyperref package 
		   \renewcommand{\backreftwosep}{ and~} % separate 2 pages
		   \renewcommand{\backreflastsep}{, and~} % separate last of longer list
		   \renewcommand*{\backref}[1]{}  % disable standard
		   \renewcommand*{\backrefalt}[4]{% detailed backref
		      \ifcase #1 %
		         \backrefnotcitedstring%
		      \or%
		         \backrefcitedsinglestring{#2}%
		      \else%
		         \backrefcitedmultistring{#2}%
		      \fi}%
}{\relax}    


% ****************************************************************
% PDF/A compliance
% ****************************************************************
% TODO not working: requires downloading color specification file in a specific
% tex folder and other hacks I don't want to spend time with
% \usepackage[a-1b]{pdfx}

% ********************************************************************
% Hyperreferences
% ********************************************************************
\hypersetup{%
    %draft,	% = no hyperlinking at all (useful in b/w printouts)
    colorlinks=true, linktocpage=true, pdfstartpage=3, pdfstartview=FitV,%
    % uncomment the following line if you want to have black links (e.g., for printing)
    %colorlinks=false, linktocpage=false, pdfborder={0 0 0}, pdfstartpage=3, pdfstartview=FitV,% 
    breaklinks=true, pdfpagemode=UseNone, pageanchor=true, pdfpagemode=UseOutlines,%
    plainpages=false, bookmarksnumbered, bookmarksopen=true, bookmarksopenlevel=1,%
    hypertexnames=true, pdfhighlight=/O,%nesting=true,%frenchlinks,%
    urlcolor=webbrown, linkcolor=RoyalBlue, citecolor=webgreen, %pagecolor=RoyalBlue,%
    %urlcolor=Black, linkcolor=Black, citecolor=Black, %pagecolor=Black,%
} 

    %pdftitle={\myTitle},%
    %pdfauthor={\textcopyright\ \myFirstAuthorName and \mySecondAuthorName, \myUni, \myFaculty},%
    %pdfsubject={},%
    %pdfkeywords={},%
    %pdfcreator={pdfLaTeX},%
    %pdfproducer={LaTeX with hyperref and classicthesis}%

%}   

% ********************************************************************
% Setup autoreferences
% ********************************************************************
% There are some issues regarding autorefnames
% http://www.ureader.de/msg/136221647.aspx
% http://www.tex.ac.uk/cgi-bin/texfaq2html?label=latexwords
% you have to redefine the makros for the 
% language you use, e.g., american, ngerman
% (as chosen when loading babel/AtBeginDocument)
% ********************************************************************
\makeatletter
\@ifpackageloaded{babel}%
    {%
       \addto\extrasamerican{%
					\renewcommand*{\figureautorefname}{Figure}%
					\renewcommand*{\tableautorefname}{Table}%
					\renewcommand*{\partautorefname}{Part}%
					\renewcommand*{\chapterautorefname}{Chapter}%
					\renewcommand*{\sectionautorefname}{Section}%
					\renewcommand*{\subsectionautorefname}{Section}%
					\renewcommand*{\subsubsectionautorefname}{Section}% 	
				}%
       \addto\extrasitalian{% 
					\renewcommand*{\paragraphautorefname}{Paragrafo}%
					\renewcommand*{\subparagraphautorefname}{Paragrafo}%
					\renewcommand*{\footnoteautorefname}{Nota a pié di pagina}%
					\renewcommand*{\FancyVerbLineautorefname}{Zeile}%
					\renewcommand*{\theoremautorefname}{Teorema}%
					\renewcommand*{\appendixautorefname}{Appendice}%
					\renewcommand*{\equationautorefname}{Equazione}%        
					\renewcommand*{\itemautorefname}{Punto}%
				}%	
			% Fix to getting autorefs for subfigures right (thanks to Belinda Vogt for changing the definition)
			\providecommand{\subfigureautorefname}{\figureautorefname}%  			
    }{\relax}
\makeatother

% ****************************************************************
% 7. Last calls before the bar closes
% ****************************************************************

\usepackage{classicthesis} 

% ****************************************************************
% ****************************************************************
% If you like the classicthesis, then I would appreciate a
% postcard. My address can be found in the file
% ClassicThesis.pdf. A collection of the postcards I received so
% far is available online at http://postcards.miede.de
% ****************************************************************

% ****************************************************************
% 1. Configure classicthesis for your needs here, e.g., remove
% "drafting" below in order to deactivate the time-stamp on the
% pages
% ****************************************************************

% !TEX root = ClassicThesis_DEIB.tex
\PassOptionsToPackage{eulerchapternumbers,
					listings,
				 	pdfspacing,
					floatperchapter,
					%linedheaders,
				 	subfig,
					beramono,
					eulermath,
					parts}{classicthesis}										

% miei packages
\usepackage[colorinlistoftodos]{todonotes}
\usepackage{pgf}
\usepackage{tikz}
\usetikzlibrary{shapes,shapes.geometric,arrows,fit,calc,positioning,automata}
\usepackage[T1]{fontenc}

% ****************************************************************
% Triggers for this config
% **************************************************************** 
\usepackage{ifthen}
\newboolean{enable-backrefs} % enable backrefs in the bibliography
\setboolean{enable-backrefs}{false} % true false
% TODO backref is incompatible?
% ****************************************************************


% ****************************************************************
% 2. Personal data and user ad-hoc commands
% ****************************************************************
\newcommand{\myTitle}{A template for master thesis at DEIB\xspace}
\newcommand{\myTitleIT}{Un modello per tesi di laurea magistrale al DEIB \xspace}
\newcommand{\myFirstAuthorName}{Giovanni Beri\xspace}
\newcommand{\myMatrFirstAuthor}{852984\xspace}
\newcommand{\mySecondAuthorName}{Emanuele Mason\xspace}
\newcommand{\myMatrSecondAuthor}{222222\xspace}
\newcommand{\mySupervisor}{Prof. Matteo Matteucci\xspace} % relatore
\newcommand{\myOtherSupervisor}{????\xspace} % co relatori
\newcommand{\myOtherOtherSupervisor}{Prof. my ASP supervisor\xspace}
\newcommand{\myCoExaminer}{Prof. Rodolfo Soncini-Sessa\xspace} % contro-relatore
\newcommand{\myFaculty}{Facolt\`a di Ingegneria\xspace}
\newcommand{\mySchool}{Scuola di Ingegneria Industriale e dell'Informazione\xspace}
\newcommand{\myDepartment}{Dipartimento di Elettronica, Informazione e Bioingegneria\xspace}
\newcommand{\myCourseFirstPart}{Master of Science in\xspace}
\newcommand{\myCourseFirstPartIT}{Corso di Laurea Magistrale in\xspace}
\newcommand{\myCourseSecondPart}{Computer Science and Engineering\xspace}
\newcommand{\myCourseSecondPartIT}{Ingegneria Informatica\xspace}
\newcommand{\myUni}{Politecnico di Milano\xspace}
\newcommand{\myLocation}{Milan\xspace}
\newcommand{\myTime}{2013\xspace}
\newcommand{\myVersion}{version 1.0\xspace}
\newcommand{\myAcademicYear}{Academic Year 2016-2017\xspace}
\newcommand{\myAcademicYearIT}{Anno Accademico 2016-2017\xspace}

% ********************************************************************
% Setup, fine tuning, and useful commands
% ********************************************************************

\newcounter{dummy} % necessary for correct hyperlinks (to index, bib, etc.)
\newlength{\abcd} % for ab..z string length calculation
\providecommand{\mLyX}{L\kern-.1667em\lower.25em\hbox{Y}\kern-.125emX\@}
% from here till the end of the section, you can modify whatever you want
\newcommand{\ie}{i.\,e.\ }
\newcommand{\Ie}{I.\,e.\ }
\newcommand{\eg}{e.\,g.\ }
\newcommand{\Eg}{E.\,g.\ }
% referencing commands
\newcommand{\myEq}[1]{equation \eqref{#1}}
\newcommand{\MyEq}[1]{Equation \eqref{#1}}
\newcommand{\myFig}[1]{figure \ref{#1}}
\newcommand{\MyFig}[1]{Figure \ref{#1}}
\newcommand{\myTab}[1]{table \ref{#1}}
\newcommand{\MyTab}[1]{Table \ref{#1}}
\newcommand{\mySubsec}[1]{subsection \ref{#1}}
\newcommand{\MySubsec}[1]{Subsection \ref{#1}}
\newcommand{\mySec}[1]{section \ref{#1}}
\newcommand{\MySec}[1]{Section \ref{#1}}
\newcommand{\myChap}[1]{chapter \ref{#1}}
\newcommand{\MyChap}[1]{Chapter \ref{#1}}
\newcommand{\myAppendix}[1]{appendix \ref{#1}}
\newcommand{\MyAppendix}[1]{Appendix \ref{#1}}
\newcommand{\myEmph}[1]{\textsc{#1}}

% **********************************************************************


% **********************************************************************
% 3. Loading some handy packages
% **********************************************************************
\PassOptionsToPackage{T1}{fontenc} % T2A for cyrillics
	\usepackage{fontenc} 
\PassOptionsToPackage{utf8}{inputenc} % latin9 (ISO-8859-9) = latin1+"Euro sign"
	\usepackage{inputenc}				

\PassOptionsToPackage{autostyle,italian=guillemets,threshold=2}{csquotes}
 	\usepackage{csquotes}

\PassOptionsToPackage{american,italian}{babel}
	 \usepackage{babel}

\usepackage{textcomp}% fix warning with missing font shapes
\usepackage{scrhack}% fix warnings when using KOMA with listings package          
\usepackage{xspace}% to get the spacing after macros right  
\usepackage{mparhack}% get marginpar right
\usepackage{fixltx2e}% fixes some LaTeX stuff
\usepackage{microtype}
\usepackage[normalem]{ulem} % to have strikethrough text

\PassOptionsToPackage{printonlyused,smaller}{acronym}
	\usepackage{acronym} % nice macros for handling all acronyms in the thesis
% **********************************************************************

%*********************************************************************************
% 3.a Math
%*********************************************************************************
\PassOptionsToPackage{fleqn}{amsmath} % math environments and more by the AMS 
 	\usepackage{amsmath}
 
\usepackage{amsthm}
\usepackage{amssymb}
\renewcommand\qedsymbol{$\blacksquare$}

\theoremstyle{definition}
\newtheorem{definition}{Definition}[chapter]
\newtheorem{theorem}{Theorem}[chapter]

\theoremstyle{plain}
\newtheorem{observation}[definition]{Observation}
\newtheorem{corollary}[theorem]{Corollary}
\newtheorem{lemma}[theorem]{Lemma}
% **********************************************************************


% **********************************************************************
% 4. Setup floats: tables, (sub)figures, and captions
% **********************************************************************
\usepackage{tabularx} % better tables
	\setlength{\extrarowheight}{3pt} % increase table row height
\newcommand{\tableheadline}[2]{\multicolumn{1}{#1}{\normalsize\spacedlowsmallcaps{#2}}}
\newcommand{\tableheadlineMore}[3]{\multicolumn{#1}{#2}{\normalsize\spacedlowsmallcaps{#3}}}
\newcommand{\tablefirstcol}[2]{\multicolumn{1}{#1}{\textbf{#2}}}

\usepackage{caption}
	\captionsetup{format=hang,labelfont={sf,bf},font=small}
\usepackage{colortbl}
\usepackage{multirow}

\usepackage{subfig}
\usepackage{siunitx}
% *********************************************************************

% *********************************************************************
% 5. Setup code listings
% *********************************************************************
\usepackage{listings}
\lstloadlanguages{bash, C++, Java, Matlab}

% for special keywords
\lstset{language=[LaTeX]Tex,
    keywordstyle=\color{RoyalBlue},%\bfseries,
    basicstyle=\small\ttfamily,
    %identifierstyle=\color{NavyBlue},
    commentstyle=\color{Green}\ttfamily,
    stringstyle=\rmfamily,
    numbers=none,%left,%
    numberstyle=\scriptsize,%\tiny
    stepnumber=5,
    numbersep=8pt,
    showstringspaces=false,
    breaklines=true,
    frameround=ftff,
    frame=single,
    belowcaptionskip=.75\baselineskip
    %frame=L
} 
% *********************************************************************


% *********************************************************************
% 6. PDFLaTeX, hyperreferences and citation backreferences
% *********************************************************************
% ********************************************************************
% Using PDFLaTeX
% ********************************************************************
\PassOptionsToPackage{pdftex,hyperfootnotes=false,pdfpagelabels}{hyperref}
	\usepackage{hyperref}  % backref linktocpage pagebackref
\pdfcompresslevel=9
\pdfadjustspacing=1 
\PassOptionsToPackage{pdftex}{graphicx}
	\usepackage{graphicx} 

% ********************************************************************
% Setup the style of the backrefs from the bibliography
% (translate the options to any language you use)
% ********************************************************************
\newcommand{\backrefnotcitedstring}{\relax}%(Not cited.)
\newcommand{\backrefcitedsinglestring}[1]{(Cited on page~#1.)}
\newcommand{\backrefcitedmultistring}[1]{(Cited on pages~#1.)}
\ifthenelse{\boolean{enable-backrefs}}%
{%
		\PassOptionsToPackage{hyperpageref}{backref}
		\usepackage{backref} % to be loaded after hyperref package 
		   \renewcommand{\backreftwosep}{ and~} % separate 2 pages
		   \renewcommand{\backreflastsep}{, and~} % separate last of longer list
		   \renewcommand*{\backref}[1]{}  % disable standard
		   \renewcommand*{\backrefalt}[4]{% detailed backref
		      \ifcase #1 %
		         \backrefnotcitedstring%
		      \or%
		         \backrefcitedsinglestring{#2}%
		      \else%
		         \backrefcitedmultistring{#2}%
		      \fi}%
}{\relax}    


% ****************************************************************
% PDF/A compliance
% ****************************************************************
% TODO not working: requires downloading color specification file in a specific
% tex folder and other hacks I don't want to spend time with
% \usepackage[a-1b]{pdfx}

% ********************************************************************
% Hyperreferences
% ********************************************************************
\hypersetup{%
    %draft,	% = no hyperlinking at all (useful in b/w printouts)
    colorlinks=true, linktocpage=true, pdfstartpage=3, pdfstartview=FitV,%
    % uncomment the following line if you want to have black links (e.g., for printing)
    %colorlinks=false, linktocpage=false, pdfborder={0 0 0}, pdfstartpage=3, pdfstartview=FitV,% 
    breaklinks=true, pdfpagemode=UseNone, pageanchor=true, pdfpagemode=UseOutlines,%
    plainpages=false, bookmarksnumbered, bookmarksopen=true, bookmarksopenlevel=1,%
    hypertexnames=true, pdfhighlight=/O,%nesting=true,%frenchlinks,%
    urlcolor=webbrown, linkcolor=RoyalBlue, citecolor=webgreen, %pagecolor=RoyalBlue,%
    %urlcolor=Black, linkcolor=Black, citecolor=Black, %pagecolor=Black,%
} 

    %pdftitle={\myTitle},%
    %pdfauthor={\textcopyright\ \myFirstAuthorName and \mySecondAuthorName, \myUni, \myFaculty},%
    %pdfsubject={},%
    %pdfkeywords={},%
    %pdfcreator={pdfLaTeX},%
    %pdfproducer={LaTeX with hyperref and classicthesis}%

%}   

% ********************************************************************
% Setup autoreferences
% ********************************************************************
% There are some issues regarding autorefnames
% http://www.ureader.de/msg/136221647.aspx
% http://www.tex.ac.uk/cgi-bin/texfaq2html?label=latexwords
% you have to redefine the makros for the 
% language you use, e.g., american, ngerman
% (as chosen when loading babel/AtBeginDocument)
% ********************************************************************
\makeatletter
\@ifpackageloaded{babel}%
    {%
       \addto\extrasamerican{%
					\renewcommand*{\figureautorefname}{Figure}%
					\renewcommand*{\tableautorefname}{Table}%
					\renewcommand*{\partautorefname}{Part}%
					\renewcommand*{\chapterautorefname}{Chapter}%
					\renewcommand*{\sectionautorefname}{Section}%
					\renewcommand*{\subsectionautorefname}{Section}%
					\renewcommand*{\subsubsectionautorefname}{Section}% 	
				}%
       \addto\extrasitalian{% 
					\renewcommand*{\paragraphautorefname}{Paragrafo}%
					\renewcommand*{\subparagraphautorefname}{Paragrafo}%
					\renewcommand*{\footnoteautorefname}{Nota a pié di pagina}%
					\renewcommand*{\FancyVerbLineautorefname}{Zeile}%
					\renewcommand*{\theoremautorefname}{Teorema}%
					\renewcommand*{\appendixautorefname}{Appendice}%
					\renewcommand*{\equationautorefname}{Equazione}%        
					\renewcommand*{\itemautorefname}{Punto}%
				}%	
			% Fix to getting autorefs for subfigures right (thanks to Belinda Vogt for changing the definition)
			\providecommand{\subfigureautorefname}{\figureautorefname}%  			
    }{\relax}
\makeatother

% ****************************************************************
% 7. Last calls before the bar closes
% ****************************************************************

\usepackage{classicthesis} 

% ****************************************************************