% !TEX root = ../ClassicThesis_DEIB.tex
%*******************************************************
% Abstract
%*******************************************************
%\renewcommand{\abstractname}{Abstract}
\addcontentsline{toc}{chapter}{\abstractname}

\pdfbookmark[1]{Abstract}{Abstract}
\begingroup
\let\clearpage\relax
\let\cleardoublepage\relax
\let\cleardoublepage\relax

\chapter*{Abstract}
This thesis work is placed in the context of \ac{GRAPE} project, an agricultural robotics project launched by \ac{ECHORD++}, that aims to the creation of an autonomous mobile manipulator for automatic deployment of synthetic pheromones dispensers in the vineyards. These devices are used in vineyards for pest management techniques, as mating disruption. The team that carried out the project development included people from both  Politecnico and Eurecat, a Catalan research center.
\par The first part of the work focused on the design of the overall software architecture, together with the identification of subtasks and the interfaces between the software modules that implement them. 
\par The thesis work proceeded with the development of odometry system, autonomous navigation, and part of the manipulation components for the target platform. The challenges induced by the unstructured nature of the vineyard environment led, after several refining steps, to the creation of a sensor fusion system that exploits \ac{IMU}, GPS, and wheels velocities to estimate the robot pose, and a mapless autonomous navigation system. On the other hand, the development of the manipulation part was complicated by the objective flaws of the robotic arm we had at disposal, that we override using techniques based on inversion of differential kinematics, including image-based visual servoing control.
\par The system also underwent two one-week validation sessions on the field, carried out in vineyards in Casciano Terme (IT) and Garriguella (ES); during these sessions, we were able to  both make the system more robust with respect to unexpected complications, and measure the performance of the system in its actual environment.

\vfill
\newpage
\pdfbookmark[1]{Sommario}{Sommario}
\chapter*{Sommario}
Questo lavoro di tesi è da collocare nel contesto del progetto \ac{GRAPE}, un progetto di robotica agricola bandito da \ac{ECHORD++} che mira alla creazione di un manipolatore mobile autonomo per la distribuzione automatica di erogatori di ferormoni sintetici all'interno dei vigneti. Erogatori di questo tipo vengono impiegati in alcune tecniche di disinfestazione, dette di "confusione sessuale".
Il team di sviluppo del progetto \ac{GRAPE} era composto sia da membri del Politecnico di Milano, sia da membri di Eurecat, un centro tecnologico catalano.
\par Inizialmente il lavoro di tesi si è focalizzato sulla progettazione dell'architettura software complessiva del sistema, attraverso \\l'identificazione di sottoproblemi e la definizione delle interfacce  tra i diversi moduli software corrispondenti.
\par Il passo successivo è stato la realizzazione del sistema odometrico, del sistema di navigazione autonoma, e di parte della logica del manipolatore per la piattaforma in oggetto. 
La struttura geometricamente indefinita del terreno e degli ostacoli nell'ambiente del vigneto ha portato, dopo svariati perfezionamenti, a un sistema di fusione sensoriale basato sull'utilizzo di \ac{IMU}, GPS, e velocità delle ruote per la stima di posizione e orientamento del robot nel tempo, e un sistema di navigazione autonoma svincolato dall'uso di mappe. Lo sviluppo della logica del manipolatore, invece, è stato fortemente influenzato dagli oggettivi limiti di funzionalità del braccio robotico in dotazione. Per aggirarli sono state sviluppate diverse tecniche basate sull'inversione della cinematica differenziale, tra cui controllo tramite asservimento visivo basato sull'immagine.
\par Il sistema è stato anche sottoposto a due sessioni di validazione sul campo, da una settimana l'una: la prima a Casciano Terme (IT), la seconda a Garriguella (ES), in cui il sistema è stato reso più robusto alle complicanze non previste in laboratorio, e le performance della piattaforma sono state testate nell'effettivo ambiente di utlizzo.

\endgroup











