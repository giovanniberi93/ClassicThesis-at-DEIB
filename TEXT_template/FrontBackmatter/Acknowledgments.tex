% !TEX root = ../ClassicThesis_DEIB.tex
%*******************************************************
% Acknowledgments
%*******************************************************
\pdfbookmark[1]{Acknowledgments}{acknowledgments}

\bigskip

\begingroup
\let\clearpage\relax
\let\cleardoublepage\relax
\let\cleardoublepage\relax
\chapter*{Ringraziamenti}
Per prima cosa voglio ringraziare il professor Matteucci, per la disponibilità e umanità con cui mi ha seguito in questo percorso di tesi; inoltre lo ringrazio di avermi dato la possibilità di lavorare sulle problematiche d'insieme del robot sviluppato, e non solo sui singoli dettagli.
\par Vorrei poi ringraziare insieme a lui le altre persone che hanno contribuito direttamente a questo lavoro di tesi: il professor Bascetta, per le preziose competenze e per l'umiltà, che hanno contribuito molto a creare un clima di collaborazione, soprattutto nelle trasferte; Gianluca Bardaro, per i centratissimi consigli tecnici, la pazienza mentre mi ingarbugliavo nel C++, e la compagnia nelle trasferte; Giulio Fontana, per il supporto tecnico e la compagnia soprattutto nelle prime disorientate settimane nell'AIRLab; Federico Polito, per l'amicizia e le battaglie contro il Kinova.
\par Per il supporto durante tutti gli anni al Poli vorrei ringraziare la mia famiglia: i miei genitori Wally e Marco, mio fratello Ciltro e mia sorella Maria Sara (che non è un granché, è vero, ma ho soltanto quella), e i nonni, zii e i cugini che mi hanno aiutato e fatto compagnia anche mentre ero a Milano. 
\par Desidero davvero ringraziare tutti gli amici, del Poli e non, con cui ho condiviso gli anni dell'università. In modi diversi sono stato aiutato da tante persone, ma dato che non riesco né a scriverle né a ricordarle tutte, esordisco con un ringraziamento plenario a tutti quelli che sanno o pensano di essere in questo elenco. Ci tengo però a specificare anche  pochi ringraziamenti particolari. Primo fra tutti, all'indissolubile gruppo degli informatici, senza cui veramente non so se ci sarei arrivato in fondo: Sid, Fatte, Beppe, Biagio, Zure e Carlos.
Poi, agli equipaggi di AppaRolle, AppaPirata (vol. I-IV) e AppaHonest, perché l'unico vantaggio dell'abitare da solo sarebbe stata la pizza tutte le sere. Agli amici della Bassa, che mi hanno sempre riaccolto anche quando ho iniziato a usare intercalari poco emiliani: Magri, Bendolo, Baraldi, Celeste, Laura, Giacomo, Co' (anche se migrata!).
\par Infine, un grazie particolare all'Anna, che mi aiuta ad andare al cuore delle cose.
\par Grazie,
\begin{flushright}
\textit{Giovanni}
\end{flushright}


\endgroup